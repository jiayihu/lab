
%% bare_conf.tex
%% V1.4
%% 2012/12/27
%% by Michael Shell
%% See:
%% http://www.michaelshell.org/
%% for current contact information.
%%
%% This is a skeleton file demonstrating the use of IEEEtran.cls
%% (requires IEEEtran.cls version 1.8 or later) with an IEEE conference paper.
%%
%% Support sites:
%% http://www.michaelshell.org/tex/ieeetran/
%% http://www.ctan.org/tex-archive/macros/latex/contrib/IEEEtran/
%% and
%% http://www.ieee.org/

%%*************************************************************************
%% Legal Notice:
%% This code is offered as-is without any warranty either expressed or
%% implied; without even the implied warranty of MERCHANTABILITY or
%% FITNESS FOR A PARTICULAR PURPOSE! 
%% User assumes all risk.
%% In no event shall IEEE or any contributor to this code be liable for
%% any damages or losses, including, but not limited to, incidental,
%% consequential, or any other damages, resulting from the use or misuse
%% of any information contained here.
%%
%% All comments are the opinions of their respective authors and are not
%% necessarily endorsed by the IEEE.
%%
%% This work is distributed under the LaTeX Project Public License (LPPL)
%% ( http://www.latex-project.org/ ) version 1.3, and may be freely used,
%% distributed and modified. A copy of the LPPL, version 1.3, is included
%% in the base LaTeX documentation of all distributions of LaTeX released
%% 2003/12/01 or later.
%% Retain all contribution notices and credits.
%% ** Modified files should be clearly indicated as such, including  **
%% ** renaming them and changing author support contact information. **
%%
%% File list of work: IEEEtran.cls, IEEEtran_HOWTO.pdf, bare_adv.tex,
%%                    bare_conf.tex, bare_jrnl.tex, bare_jrnl_compsoc.tex,
%%                    bare_jrnl_transmag.tex
%%*************************************************************************

% *** Authors should verify (and, if needed, correct) their LaTeX system  ***
% *** with the testflow diagnostic prior to trusting their LaTeX platform ***
% *** with production work. IEEE's font choices can trigger bugs that do  ***
% *** not appear when using other class files.                            ***
% The testflow support page is at:
% http://www.michaelshell.org/tex/testflow/



% Note that the a4paper option is mainly intended so that authors in
% countries using A4 can easily print to A4 and see how their papers will
% look in print - the typesetting of the document will not typically be
% affected with changes in paper size (but the bottom and side margins will).
% Use the testflow package mentioned above to verify correct handling of
% both paper sizes by the user's LaTeX system.
%
% Also note that the "draftcls" or "draftclsnofoot", not "draft", option
% should be used if it is desired that the figures are to be displayed in
% draft mode.
%
\documentclass[conference]{IEEEtran}
% Add the compsoc option for Computer Society conferences.
%
% If IEEEtran.cls has not been installed into the LaTeX system files,
% manually specify the path to it like:
% \documentclass[conference]{../sty/IEEEtran}





% Some very useful LaTeX packages include:
% (uncomment the ones you want to load)


% *** MISC UTILITY PACKAGES ***
%
%\usepackage{ifpdf}
% Heiko Oberdiek's ifpdf.sty is very useful if you need conditional
% compilation based on whether the output is pdf or dvi.
% usage:
% \ifpdf
%   % pdf code
% \else
%   % dvi code
% \fi
% The latest version of ifpdf.sty can be obtained from:
% http://www.ctan.org/tex-archive/macros/latex/contrib/oberdiek/
% Also, note that IEEEtran.cls V1.7 and later provides a builtin
% \ifCLASSINFOpdf conditional that works the same way.
% When switching from latex to pdflatex and vice-versa, the compiler may
% have to be run twice to clear warning/error messages.






% *** CITATION PACKAGES ***
%
\usepackage{cite}
% cite.sty was written by Donald Arseneau
% V1.6 and later of IEEEtran pre-defines the format of the cite.sty package
% \cite{} output to follow that of IEEE. Loading the cite package will
% result in citation numbers being automatically sorted and properly
% "compressed/ranged". e.g., [1], [9], [2], [7], [5], [6] without using
% cite.sty will become [1], [2], [5]--[7], [9] using cite.sty. cite.sty's
% \cite will automatically add leading space, if needed. Use cite.sty's
% noadjust option (cite.sty V3.8 and later) if you want to turn this off
% such as if a citation ever needs to be enclosed in parenthesis.
% cite.sty is already installed on most LaTeX systems. Be sure and use
% version 4.0 (2003-05-27) and later if using hyperref.sty. cite.sty does
% not currently provide for hyperlinked citations.
% The latest version can be obtained at:
% http://www.ctan.org/tex-archive/macros/latex/contrib/cite/
% The documentation is contained in the cite.sty file itself.






% *** GRAPHICS RELATED PACKAGES ***
%
\ifCLASSINFOpdf
  \usepackage[pdftex]{graphicx}
  % declare the path(s) where your graphic files are
  \graphicspath{{../pdf/}{../jpeg/}}
  % and their extensions so you won't have to specify these with
  % every instance of \includegraphics
  \DeclareGraphicsExtensions{.pdf,.jpeg,.png}
\else
  % or other class option (dvipsone, dvipdf, if not using dvips). graphicx
  % will default to the driver specified in the system graphics.cfg if no
  % driver is specified.
  % \usepackage[dvips]{graphicx}
  % declare the path(s) where your graphic files are
  % \graphicspath{{../eps/}}
  % and their extensions so you won't have to specify these with
  % every instance of \includegraphics
  % \DeclareGraphicsExtensions{.eps}
\fi
% graphicx was written by David Carlisle and Sebastian Rahtz. It is
% required if you want graphics, photos, etc. graphicx.sty is already
% installed on most LaTeX systems. The latest version and documentation
% can be obtained at: 
% http://www.ctan.org/tex-archive/macros/latex/required/graphics/
% Another good source of documentation is "Using Imported Graphics in
% LaTeX2e" by Keith Reckdahl which can be found at:
% http://www.ctan.org/tex-archive/info/epslatex/
%
% latex, and pdflatex in dvi mode, support graphics in encapsulated
% postscript (.eps) format. pdflatex in pdf mode supports graphics
% in .pdf, .jpeg, .png and .mps (metapost) formats. Users should ensure
% that all non-photo figures use a vector format (.eps, .pdf, .mps) and
% not a bitmapped formats (.jpeg, .png). IEEE frowns on bitmapped formats
% which can result in "jaggedy"/blurry rendering of lines and letters as
% well as large increases in file sizes.
%
% You can find documentation about the pdfTeX application at:
% http://www.tug.org/applications/pdftex





% *** MATH PACKAGES ***
%
%\usepackage[cmex10]{amsmath}
% A popular package from the American Mathematical Society that provides
% many useful and powerful commands for dealing with mathematics. If using
% it, be sure to load this package with the cmex10 option to ensure that
% only type 1 fonts will utilized at all point sizes. Without this option,
% it is possible that some math symbols, particularly those within
% footnotes, will be rendered in bitmap form which will result in a
% document that can not be IEEE Xplore compliant!
%
% Also, note that the amsmath package sets \interdisplaylinepenalty to 10000
% thus preventing page breaks from occurring within multiline equations. Use:
%\interdisplaylinepenalty=2500
% after loading amsmath to restore such page breaks as IEEEtran.cls normally
% does. amsmath.sty is already installed on most LaTeX systems. The latest
% version and documentation can be obtained at:
% http://www.ctan.org/tex-archive/macros/latex/required/amslatex/math/





% *** SPECIALIZED LIST PACKAGES ***
%
%\usepackage{algorithmic}
% algorithmic.sty was written by Peter Williams and Rogerio Brito.
% This package provides an algorithmic environment fo describing algorithms.
% You can use the algorithmic environment in-text or within a figure
% environment to provide for a floating algorithm. Do NOT use the algorithm
% floating environment provided by algorithm.sty (by the same authors) or
% algorithm2e.sty (by Christophe Fiorio) as IEEE does not use dedicated
% algorithm float types and packages that provide these will not provide
% correct IEEE style captions. The latest version and documentation of
% algorithmic.sty can be obtained at:
% http://www.ctan.org/tex-archive/macros/latex/contrib/algorithms/
% There is also a support site at:
% http://algorithms.berlios.de/index.html
% Also of interest may be the (relatively newer and more customizable)
% algorithmicx.sty package by Szasz Janos:
% http://www.ctan.org/tex-archive/macros/latex/contrib/algorithmicx/




% *** ALIGNMENT PACKAGES ***
%
%\usepackage{array}
% Frank Mittelbach's and David Carlisle's array.sty patches and improves
% the standard LaTeX2e array and tabular environments to provide better
% appearance and additional user controls. As the default LaTeX2e table
% generation code is lacking to the point of almost being broken with
% respect to the quality of the end results, all users are strongly
% advised to use an enhanced (at the very least that provided by array.sty)
% set of table tools. array.sty is already installed on most systems. The
% latest version and documentation can be obtained at:
% http://www.ctan.org/tex-archive/macros/latex/required/tools/


% IEEEtran contains the IEEEeqnarray family of commands that can be used to
% generate multiline equations as well as matrices, tables, etc., of high
% quality.




% *** SUBFIGURE PACKAGES ***
%\ifCLASSOPTIONcompsoc
%  \usepackage[caption=false,font=normalsize,labelfont=sf,textfont=sf]{subfig}
%\else
%  \usepackage[caption=false,font=footnotesize]{subfig}
%\fi
% subfig.sty, written by Steven Douglas Cochran, is the modern replacement
% for subfigure.sty, the latter of which is no longer maintained and is
% incompatible with some LaTeX packages including fixltx2e. However,
% subfig.sty requires and automatically loads Axel Sommerfeldt's caption.sty
% which will override IEEEtran.cls' handling of captions and this will result
% in non-IEEE style figure/table captions. To prevent this problem, be sure
% and invoke subfig.sty's "caption=false" package option (available since
% subfig.sty version 1.3, 2005/06/28) as this is will preserve IEEEtran.cls
% handling of captions.
% Note that the Computer Society format requires a larger sans serif font
% than the serif footnote size font used in traditional IEEE formatting
% and thus the need to invoke different subfig.sty package options depending
% on whether compsoc mode has been enabled.
%
% The latest version and documentation of subfig.sty can be obtained at:
% http://www.ctan.org/tex-archive/macros/latex/contrib/subfig/




% *** FLOAT PACKAGES ***
%
%\usepackage{fixltx2e}
% fixltx2e, the successor to the earlier fix2col.sty, was written by
% Frank Mittelbach and David Carlisle. This package corrects a few problems
% in the LaTeX2e kernel, the most notable of which is that in current
% LaTeX2e releases, the ordering of single and double column floats is not
% guaranteed to be preserved. Thus, an unpatched LaTeX2e can allow a
% single column figure to be placed prior to an earlier double column
% figure. The latest version and documentation can be found at:
% http://www.ctan.org/tex-archive/macros/latex/base/


\usepackage{stfloats}
% stfloats.sty was written by Sigitas Tolusis. This package gives LaTeX2e
% the ability to do double column floats at the bottom of the page as well
% as the top. (e.g., "\begin{figure*}[!b]" is not normally possible in
% LaTeX2e). It also provides a command:
%\fnbelowfloat
% to enable the placement of footnotes below bottom floats (the standard
% LaTeX2e kernel puts them above bottom floats). This is an invasive package
% which rewrites many portions of the LaTeX2e float routines. It may not work
% with other packages that modify the LaTeX2e float routines. The latest
% version and documentation can be obtained at:
% http://www.ctan.org/tex-archive/macros/latex/contrib/sttools/
% Do not use the stfloats baselinefloat ability as IEEE does not allow
% \baselineskip to stretch. Authors submitting work to the IEEE should note
% that IEEE rarely uses double column equations and that authors should try
% to avoid such use. Do not be tempted to use the cuted.sty or midfloat.sty
% packages (also by Sigitas Tolusis) as IEEE does not format its papers in
% such ways.
% Do not attempt to use stfloats with fixltx2e as they are incompatible.
% Instead, use Morten Hogholm'a dblfloatfix which combines the features
% of both fixltx2e and stfloats:
%
% \usepackage{dblfloatfix}
% The latest version can be found at:
% http://www.ctan.org/tex-archive/macros/latex/contrib/dblfloatfix/




% *** PDF, URL AND HYPERLINK PACKAGES ***
%
\usepackage{url}
% url.sty was written by Donald Arseneau. It provides better support for
% handling and breaking URLs. url.sty is already installed on most LaTeX
% systems. The latest version and documentation can be obtained at:
% http://www.ctan.org/tex-archive/macros/latex/contrib/url/
% Basically, \url{my_url_here}.




% *** Do not adjust lengths that control margins, column widths, etc. ***
% *** Do not use packages that alter fonts (such as pslatex).         ***
% There should be no need to do such things with IEEEtran.cls V1.6 and later.
% (Unless specifically asked to do so by the journal or conference you plan
% to submit to, of course. )


% correct bad hyphenation here
\hyphenation{op-tical net-works semi-conduc-tor}


\begin{document}
%
% paper title
% can use linebreaks \\ within to get better formatting as desired
% Do not put math or special symbols in the title.
\title{Data management patterns in \\microservices architecture}


% author names and affiliations
% use a multiple column layout for up to three different
% affiliations
\author{\IEEEauthorblockN{Giovanni Jiayi Hu}
\IEEEauthorblockA{Department of Mathematics\\
University of Padua, Italy I-35121\\
Email: giovannijiayi.hu@studenti.unipd.it}}

% conference papers do not typically use \thanks and this command
% is locked out in conference mode. If really needed, such as for
% the acknowledgment of grants, issue a \IEEEoverridecommandlockouts
% after \documentclass

% for over three affiliations, or if they all won't fit within the width
% of the page, use this alternative format:
% 
%\author{\IEEEauthorblockN{Michael Shell\IEEEauthorrefmark{1},
%Homer Simpson\IEEEauthorrefmark{2},
%James Kirk\IEEEauthorrefmark{3}, 
%Montgomery Scott\IEEEauthorrefmark{3} and
%Eldon Tyrell\IEEEauthorrefmark{4}}
%\IEEEauthorblockA{\IEEEauthorrefmark{1}School of Electrical and Computer Engineering\\
%Georgia Institute of Technology,
%Atlanta, Georgia 30332--0250\\ Email: see http://www.michaelshell.org/contact.html}
%\IEEEauthorblockA{\IEEEauthorrefmark{2}Twentieth Century Fox, Springfield, USA\\
%Email: homer@thesimpsons.com}
%\IEEEauthorblockA{\IEEEauthorrefmark{3}Starfleet Academy, San Francisco, California 96678-2391\\
%Telephone: (800) 555--1212, Fax: (888) 555--1212}
%\IEEEauthorblockA{\IEEEauthorrefmark{4}Tyrell Inc., 123 Replicant Street, Los Angeles, California 90210--4321}}




% use for special paper notices
%\IEEEspecialpapernotice{(Invited Paper)}




% make the title area
\maketitle

% As a general rule, do not put math, special symbols or citations
% in the abstract
\begin{abstract}
A microservice architecture is a service-oriented architecture composed of loosely coupled elements that have bounded contexts. Each service has a focused, cohesive set of responsibilities.
It comes then with no surprise that a key characteristic of the architecture is that the services are loosely coupled and communicate only via APIs. To achieve loose coupling, each service is required to to have its own datastore. However, this introduces a new range of obstacles.
We must, in fact, implement transactions that work across multiple distributed services. An operation that spans services must indeed use what’s known as a saga, a message-driven sequence of local transactions, to maintain data consistency. However, sagas lack the isolation feature of traditional ACID transactions. As a result, an application must use what are known as countermeasures, design techniques that prevent or reduce the impact of concurrency anomalies caused by the lack of isolation.
On top of that, writing queries in a microservice architecture reveals to be likewise challenging. Implementing queries in the existing monolithic application is relatively straightforward because it has a single database, but in a microservice architecture queries often need to retrieve data that are scattered among the databases owned by multiple services. Two different patterns are presented for implementing query operations: the more straightforward API composition pattern and the more powerful Command query responsibility segregation (CQRS) pattern.
\end{abstract}

% no keywords




% For peer review papers, you can put extra information on the cover
% page as needed:
% \ifCLASSOPTIONpeerreview
% \begin{center} \bfseries EDICS Category: 3-BBND \end{center}
% \fi
%
% For peerreview papers, this IEEEtran command inserts a page break and
% creates the second title. It will be ignored for other modes.
\IEEEpeerreviewmaketitle



\section{Introduction}
A microservice is as a small self-contained application that has a single responsibility, a lightweight stack, and can be deployed, scaled and tested independently. \cite{exploring-microservices} The central aspect of this definition is the independence, the ability for a service to change without affecting any other.
In order to achieve this praised independence, the database architecture must respect the following requirements:
\begin{itemize}
  \item Services must be loosely coupled so that they can be developed, deployed and scaled independently
  \item Some business transactions must enforce invariants that span multiple services. For instance, before placing an order you must verify that a new order will not exceed the customer’s credit limit.
  \item Some business transactions need to query data that is owned by multiple services. For example, displying the orders require querying the Orders service to gather the cost and the Restaurant service to know if an order has been accepted
  \item Databases must be replicated and sharded in order to scale.
  \item Different services have different data storage requirements. For some services, a relational database is the best choice. Other services might need a NoSQL database such as MongoDB, which is good at storing complex, unstructured data.
\end{itemize}

A common solution to these requirements is to keep each microservice’s persistent data private to that service and accessible only via its API. The transactions of a service involve only its database. 
Using a database per service ensures that the services are loosely coupled and guarantees higher resilience. Changes to one service’s database does not impact any other services. Besides each service can use the type of database that is best suited to its needs. For example, a service that does text searches could use ElasticSearch. \cite{database-per-service} 

However using a database per service introduces severe drawbacks which must be carefully evaluated: implementing business transactions that span multiple services is no more straightforward and  implementing queries which join data that is now in multiple databases is challenging.

\section{Distributed transactions}

ACID (Atomicity, Consistency, Isolation, Durability) transactions greatly simplify the job of the developer by providing the illusion that each transaction has exclusive access to the data. In a microservice architecture, transactions that are within a single service can still use ACID transactions. The challenge, however, lies in implementing transactions for operations that update data owned by multiple services.

In many ways, the biggest obstacle that developers will face when adopting microservices is moving from a single database with ACID transactions to a multi-database architecture with ACD sagas. They’re used to the simplicity of the ACID transaction model. \cite{microservices-patterns}

The traditional approach to maintaining data consistency across multiple services, databases, or message brokers is to use distributed transactions. The de facto standard for distributed transaction management is the X/Open Distributed Transaction Processing (DTP) Model\footnote{\url{https://en.wikipedia.org/wiki/X/Open_XA}}, which uses two-phase commit (2PC) to ensure that all participants in a transaction either commit or rollback. 

Several problems arise however when a 2PC protocol is used in a system where failures occur. For instance, the coordinator as well as the participants have states in which they block waiting for incoming messages. \cite{distributed-systems}

Nevertheless the main problem with traditional distributed transactions is that they reduce availability. In order for a distributed transaction to commit, all the participating services must be available.

The availability is the product of the availability of all of the participants in the transaction. If a distributed transaction involves two services that are 99.5\% available, then the overall availability is 99\%, which is significantly less. Each additional service involved in a distributed transaction further reduces availability.

On top of that, the famous CAP theorem provides another proof of the unsuitability of distributed transactions:

\hfill It is impossible for a distributed data store to simultaneously provide more than two out of the following three guarantees: consistency, availability, partition tolerance.
 
\hfill Eric Brewer

Many systems nowadays drop consistency in favor of availability, settling for eventual consistency. Inconsistency can be tolerated for two reasons: for improving read and write performance under highly concurrent conditions and for handling partition cases where a majority model would render part of the system unavailable even though the nodes are up and running. \cite{consistency-vs-availability}

It is clear at this point that to solve the complex problem of maintaining data consistency in a microservice architecture, an application must use a different mechanism that builds on top of the concept of loosely coupled, asynchronous services. This is where the Saga pattern comes in.

\section{The Saga pattern}

A saga is a sequence of local transactions. Each local transaction updates data within a single service using the familiar ACID transaction.

A saga is initiated by a command event and it reacts to subsequent command events, generating new ones and thus allowing command handlers to be kept independent.

While handling a request, the service doesn’t synchronously interact with any other services. Instead, it asynchronously sends messages to other services which are not required to be available at the same time. Eventually, any unavailable service will come back up and process queued messages.

\begin{figure}[!htbp]
\centering
\includegraphics[width=3in]{jpeg/order-saga}
\caption{Creating an Order using a saga}
\label{order_saga}
\end{figure}

The example saga used throughout this section is the \texttt{Create Order Saga}, which is shown in Figure \ref{order_saga}. The saga’s first local transaction is initiated by a command event to create an order. The other five local transactions are each triggered by command events signaling the completion of the previous one.

This saga consists of the following local transactions:

\begin{enumerate}
  \item Order Service: Create an Order in an \texttt{APPROVAL\_PENDING} state;
  \item Consumer Service: Verify that the consumer can place an order;
  \item Kitchen Service: Validate order details and create a Ticket in the CREATE \texttt{\_PENDING};
  \item Accounting Service: Authorize consumer’s credit card;
  \item Kitchen Service: Change the state of the Ticket to \texttt{AWAITING\_ACCEPTANCE};
  \item Order Service: Change the state of the Order to APPROVED.
\end{enumerate}

\subsection{Compensating transactions}

A great feature of traditional ACID transactions is that the business logic can easily roll back a transaction if it detects the violation of a business rule. Unfortunately, sagas can’t be automatically rolled back, because each step commits its changes to the local database. 

To see how compensating transactions are used, imagine a scenario where the authorization of the consumer’s credit card fails. In this scenario, the saga executes the following local transactions:

\begin{enumerate}
  \item Order Service: Create an Order in an \texttt{APPROVAL\_PENDING} state;
  \item Consumer Service: Verify that the consumer can place an order;
  \item Kitchen Service: Validate order details and create a Ticket in the \texttt{CREATE\_PENDING} state;
  \item Accounting Service: Authorize consumer’s credit card, which fails;
  \item Kitchen Service: Change the state of the Ticket to \texttt{CREATE\_REJECTE}D;
  \item Order Service: Change the state of the Order to \texttt{REJECTED}.
\end{enumerate}

The fifth and sixth steps are compensating transactions that undo the updates made by Kitchen Service and Order Service, respectively.

A saga’s coordination logic is responsible for sequencing the execution of forward and compensating transactions. \cite{microservices-patterns}

\subsection{Lack of isolation}

The I in ACID stands for isolation. The isolation property of ACID transactions ensures that the outcome of executing multiple transactions concurrently is the same as if they were executed in some serial order. The database provides the illusion that each ACID transaction has exclusive access to the data.

The challenge with using sagas is that they lack the isolation property of ACID transactions. That’s because the updates made by each of a saga’s local transactions are immediately visible to other sagas once that transaction commits. This behavior can cause two problems. First, other sagas can change the data accessed by the saga while it’s executing. And other sagas can read its data before the saga has completed its updates, and consequently can be exposed to inconsistent data.

This lack of isolation potentially causes what the database literature calls \textit{anomalies}. An anomaly is when a transaction reads or writes data in a way that it wouldn’t if transactions were executed one at time. When an anomaly occurs, the outcome of executing sagas concurrently is different than if they were executed serially.

The lack of isolation can cause the following three anomalies \cite{microservices-patterns}:

\begin{enumerate}
  \item \textit{Lost updates}: One saga overwrites without reading changes made by another saga;
  \item \textit{Dirty reads}: A transaction or a saga reads the updates made by a saga that has not yet completed those updates;
  \item \textit{Fuzzy/non-repeatable reads}: Two different steps of a saga read the same data and get different results because another saga has made updates;
\end{enumerate}

It’s the responsibility of the developer to adopt a set of countermeasures for handling anomalies caused by lack of isolation that either prevent one or more anomalies or minimize their impact on the business. \cite{semantic-acid}

\subsection{Reliable events}

It’s essential that the database update and the publishing of the event happen atomically. Consequently, to communicate reliably, the saga participants must use transactional messaging. The second issue you need to consider is ensuring that a saga participant must be able to map each event that it receives to its own data. The solution is for a saga participant to publish events containing a correlation id, which is data that enables other participants to perform the mapping.

\section{Communication}

The microservice architecture structures an application as a set of services which must often collaborate in order to handle a request. Because service instances are typically processes running on multiple machines, they must interact using Inter-Process Communication (IPC).

The choice of IPC mechanism is an important architectural decision since it can impact application availability and it intersects with transaction management. As we have seen with the saga pattern, we favor loosely coupled services that communicate with one another using asynchronous messaging.

Synchronous protocols such as REST and RPC are used mostly to communicate with other applications. The problem with REST is that it’s a synchronous protocol: an HTTP client must wait for the service to send a response. Whenever services communicate using a synchronous protocol, the availability of the application is reduced.

\subsection{Message Queuing}

The basic idea behind a message-queuing system is that applications communicate by inserting messages in specific queues. 

An important aspect of message-queuing systems is that a sender is generally given only the guarantees that its message will eventually be inserted in the recipient’s queue. No guarantees are given about when, or even if the message will actually be read, which is completely determined by the behavior of the recipient.

These semantics permit communication to be loosely coupled in time. When selecting a message broker, you have various factors to consider, including the following:

\begin{itemize}
  \item Messaging ordering: Does the message broker preserve ordering of messages?
  \item Delivery guarantees: What kind of delivery guarantees does the broker make?
  \item Persistence: Are messages persisted to disk and able to survive broker crashes?
  \item Durability: If a consumer reconnects to the message broker, will it receive the messages that were sent while it was disconnected?
  \item Scalability: How scalable is the message broker?
  \item Latency: What is the end-to-end latency?
\end{itemize}

Each broker makes different trade-offs. For example, a very low-latency broker might not preserve ordering, make no guarantees to deliver messages, and only store messages in memory. A messaging broker that guarantees delivery and reliably stores messages on disk will probably have higher latency. Which kind of message broker is the best fit depends on your application’s requirements. 

After picking the proper broker solution for your application, we can see how the saga orchestrator communicates with the participants using async reply-style interaction based on message queues. To execute a saga step, it sends a command message to a participant telling it what operation to perform. After the saga participant has performed the operation, it sends a reply message to the orchestrator. The orchestrator then processes the message and determines which saga step to perform next.

\begin{figure}[!htbp]
\centering
\includegraphics[width=3in]{jpeg/orchestrator-saga}
\caption{Order Service implements a saga orchestrator, which invokes the saga participants using asynchronous request/response.}
\label{orchestrator_saga}
\end{figure}

\section{Queries}

API composition via API Gateway: very inefficient in-memory join. Can use caching with Redis or Memcached to avoid querying other databases too often. Also more computing and network resources are required, increasing the cost of running the application. Another drawback of this pattern is reduced availability. There’s a risk also, that a query operation will return inconsistent data.
    - CQRS: Have duplicated data and use events to keep track. Drawbacks here are: you only have eventual consistency as the data duplication is asynchronous.(Better availability than consistency)

CQRS
Three problems that are commonly encountered when implementing queries in a microservice architecture:

1. Using the API composition pattern to retrieve data scattered across multiple services results in expensive, inefficient in-memory joins.
2. The service that owns the data stores the data in a form or in a database that doesn’t efficiently support the required query.
3. The need to separate concerns means that the service that owns the data isn’t the service that should implement the query operation.

- **write side** (or a command side) - The command side modules and data model implement create, update, and delete operations (CUD), ensuring business rules and handling commands. The command-side domain model handles CRUD operations and is mapped to its own database. It may also handle simple queries, such as nonjoin, primary key-based queries. **The command side publishes domain events whenever its data changes.**
- **read side** (or a query side) - takes events produced by the write side and uses them to build and maintain a model that is suitable for answering the client’s queries. The query side uses whatever kind of database makes sense for the queries that it must support. The query side has event handlers that subscribe to domain events and update the database or databases.

CQRS has both benefits and drawbacks. The benefits are as follows:

- Enables the efficient implementation of queries in a microservice architecture
- Enables the efficient implementation of diverse queries
- Makes querying possible in an event sourcing-based application
- Improves separation of concerns

Even though CQRS has several benefits, it also has significant drawbacks:

- More complex architecture
- Dealing with the **replication lag**: there’s delay between when the command side publishes an event and when that event is processed by the query side and the view updated.

A service has an API that provides its clients access to its functionality. There are two types of operations: commands and queries. The API consists of commands, queries, and events.

% An example of a floating figure using the graphicx package.
% Note that \label must occur AFTER (or within) \caption.
% For figures, \caption should occur after the \includegraphics.
% Note that IEEEtran v1.7 and later has special internal code that
% is designed to preserve the operation of \label within \caption
% even when the captionsoff option is in effect. However, because
% of issues like this, it may be the safest practice to put all your
% \label just after \caption rather than within \caption{}.
%
% Reminder: the "draftcls" or "draftclsnofoot", not "draft", class
% option should be used if it is desired that the figures are to be
% displayed while in draft mode.
%
%\begin{figure}[!t]
%\centering
%\includegraphics[width=2.5in]{myfigure}
% where an .eps filename suffix will be assumed under latex, 
% and a .pdf suffix will be assumed for pdflatex; or what has been declared
% via \DeclareGraphicsExtensions.
%\caption{Simulation Results.}
%\label{fig_sim}
%\end{figure}

% Note that IEEE typically puts floats only at the top, even when this
% results in a large percentage of a column being occupied by floats.


% An example of a double column floating figure using two subfigures.
% (The subfig.sty package must be loaded for this to work.)
% The subfigure \label commands are set within each subfloat command,
% and the \label for the overall figure must come after \caption.
% \hfil is used as a separator to get equal spacing.
% Watch out that the combined width of all the subfigures on a 
% line do not exceed the text width or a line break will occur.
%
%\begin{figure*}[!t]
%\centering
%\subfloat[Case I]{\includegraphics[width=2.5in]{box}%
%\label{fig_first_case}}
%\hfil
%\subfloat[Case II]{\includegraphics[width=2.5in]{box}%
%\label{fig_second_case}}
%\caption{Simulation results.}
%\label{fig_sim}
%\end{figure*}
%
% Note that often IEEE papers with subfigures do not employ subfigure
% captions (using the optional argument to \subfloat[]), but instead will
% reference/describe all of them (a), (b), etc., within the main caption.


% An example of a floating table. Note that, for IEEE style tables, the 
% \caption command should come BEFORE the table. Table text will default to
% \footnotesize as IEEE normally uses this smaller font for tables.
% The \label must come after \caption as always.
%
%\begin{table}[!t]
%% increase table row spacing, adjust to taste
%\renewcommand{\arraystretch}{1.3}
% if using array.sty, it might be a good idea to tweak the value of
% \extrarowheight as needed to properly center the text within the cells
%\caption{An Example of a Table}
%\label{table_example}
%\centering
%% Some packages, such as MDW tools, offer better commands for making tables
%% than the plain LaTeX2e tabular which is used here.
%\begin{tabular}{|c||c|}
%\hline
%One & Two\\
%\hline
%Three & Four\\
%\hline
%\end{tabular}
%\end{table}


% Note that IEEE does not put floats in the very first column - or typically
% anywhere on the first page for that matter. Also, in-text middle ("here")
% positioning is not used. Most IEEE journals/conferences use top floats
% exclusively. Note that, LaTeX2e, unlike IEEE journals/conferences, places
% footnotes above bottom floats. This can be corrected via the \fnbelowfloat
% command of the stfloats package.



\section{Conclusion}
The conclusion goes here.




% conference papers do not normally have an appendix





% trigger a \newpage just before the given reference
% number - used to balance the columns on the last page
% adjust value as needed - may need to be readjusted if
% the document is modified later
%\IEEEtriggeratref{8}
% The "triggered" command can be changed if desired:
%\IEEEtriggercmd{\enlargethispage{-5in}}

% references section

% can use a bibliography generated by BibTeX as a .bbl file
% BibTeX documentation can be easily obtained at:
% http://www.ctan.org/tex-archive/biblio/bibtex/contrib/doc/
% The IEEEtran BibTeX style support page is at:
% http://www.michaelshell.org/tex/ieeetran/bibtex/
%\bibliographystyle{IEEEtran}
% argument is your BibTeX string definitions and bibliography database(s)
%\bibliography{IEEEabrv,../bib/paper}
%
% <OR> manually copy in the resultant .bbl file
% set second argument of \begin to the number of references
% (used to reserve space for the reference number labels box)
\begin{thebibliography}{1}

\bibitem{exploring-microservices}
Exploring microservices
\bibitem{database-per-service}
Available: https://microservices.io/patterns/data/database-per-service.html
\bibitem{microservices-patterns}
Available: https://microservices.io/book
\bibitem{distributed-systems}
Available: https://microservices.io/book
\bibitem{consistency-vs-availability}
Available: https://www.infoq.com/news/2008/01/consistency-vs-availability
\bibitem{semantic-acid}
Available: https://dl.acm.org/citation.cfm?id=284472.284478

\end{thebibliography}




% that's all folks
\end{document}


